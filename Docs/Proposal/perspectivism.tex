\input{header.tex}


\title{
Applied Perspectivism : Proposition of an Epistemological Framework for Integrated Approaches to Complex Systems
\bigskip\\
\textit{Project Proposal}
}
\author{\textit{Authors}\medskip\\
\textit{Affiliations}
}
\date{}


\maketitle

\justify


\begin{abstract}

\end{abstract}


%%%%%%%%%%%%%%%%%%%%
\section{Introduction}

The study of complex systems requires the construction of integrated approaches as Gell-Mann points out in~\cite{}% TODO precise cit ?
 : an overspecialization in scientific fields is necessary but should not be detrimental to bridges between disciplines and interdisciplinarity that are key to capture typical features of complexity. Indeed, the European Roadmap for Complex Systems can be understood as a two dimensional matrix crossing scales with fields of study, columns consisting in integrative disciplines such as integrative biology, territorial science, % examples
 and rows of transversal questions common to all fields : emergence, scaling laws, multiscalarity, etc. % idem
 However, the practice of interdisciplinarity, despite a growing trend in recent years % cit. nature interdisc.
 is far from being natural, as many examples show % cit Dupuy Benguigui ? cit Network Science ? cit essay ?
 , for diverse reasons such as diverging objectives, formalisms, generally resulting from different conceptions of the studied systems.
 The perspectivist description of science, developed by Giere~\cite{}%
 is particularly appropriate to describe this situation : any scientific entreprise can be understood



%%%%%%%%%%%%%%%%%%%%
\section{Research objectives}


%%%%%%%%%%%%%%%%%%%%
\section{Project description}




%%%%%%%%%%%%%%%%%%%%
\section{Project organisation}











%%%%%%%%%%%%%%%%%%%%
%% Biblio
%%%%%%%%%%%%%%%%%%%%

\bibliographystyle{apalike}
\bibliography{biblio}


\end{document}


