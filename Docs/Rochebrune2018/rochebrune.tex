\input{WPHeader.tex}


\title{Applied Perspectivism to Foster Interdisciplinarity\\\medskip
\textit{Ecole thématique de Rochebrune 2018}
}
\author{\noun{Juste Raimbault}$^{1,2}$\\
$^{1}$ UMR CNRS 8504 Géographie-cités\\
$^{2}$ UMR-T IFSTTAR 9403 LVMT
}
\date{}


\maketitle

\justify

\begin{abstract}
	
\end{abstract}




\paragraph{Perspectives and Ontologies}




\paragraph{Applied Perpectivism and Model Coupling}



\paragraph{An agent-based Model of Interdisciplinarity}

The second part of this paper aims at providing quantitative evidence for the epistemological point of view described above and inform potential implementation of some of its processes. Therefore, we introduce a simple agent-based model of scientific research focusing on the interplay between disciplinary and interdisciplinary research. The rationale relies on the basic assumption that scientists can choose when starting a new project between interdisciplinary collaboration and a work within their discipline.

Agents are $N$ scientists $A_i$, characterised by a probability distribution $d(x)$ representing their disciplinary positioning in an abstract way. They also have a time budget 




%%%%%%%%%%%%%%%%%%%%
%% Biblio
%%%%%%%%%%%%%%%%%%%%
\footnotesize

\bibliographystyle{apalike}
\bibliography{/Users/Juste/ComplexSystems/CityNetwork/Biblio/Bibtex/CityNetwork,biblio}


\end{document}
