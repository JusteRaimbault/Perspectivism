\input{WPHeader.tex}


\title{An Agent-based Model of Interdisciplinary Interactions\\\medskip
\textit{Ecole thématique de Rochebrune 2018\\
Proposition de Communication
}
}
\author{\noun{Juste Raimbault}$^{1,2}$\\
$^{1}$ UMR CNRS 8504 Géographie-cités\\
$^{2}$ UMR-T IFSTTAR 9403 LVMT
}
\date{}


\maketitle

\justify

\begin{abstract}
	This paper introduces a parcimonious agent-based model of interdisciplinary relationships in collective entreprises of knowledge discovery. Under simple behavioral assumptions, we show that individual choices influence the distribution of compromise points between emergent level of disciplinary depth and interdisciplinarity, in a non-linear way. We claim that independently of each research subject, the organization of research, and more particularly the local balancing between vertical and horizontal research, strongly influence the final positioning of research results and the extent of the knowledge front, what suggests direct implications for research policies.
\end{abstract}


\cite{lariviere2010relationship} : empirical evidence of an optimal intermediate level of interdisciplinarity for impact % rq : depending on indicator, no reason to be the same ; rq2 : no citation practices in our model.



\subsection*{Perspectivism and Model Coupling}

Beyond the simplifying opposition between fully constructivist and realistic approaches to science, several alternatives have been developed, among which Perspectivism \cite{giere2010scientific} is a way to tackle most of the issues opposing these two by taking an agent-based approach to the production of scientific knowledge. The principal feature of this point of view is to consider each scientific enterprise as a single perspective, in which an agent aims to understand an aspect of the real world (the ontology) with the mean of a medium, that is considered as a model. Constituted disciplines thus contains more or less compatible perspectives. The explicitation of this approach has been done by~\cite{raimbault2017knowledge} to embed it into knowledge domains, as a generalization of knowledge domains introduced by~\cite{livet2010}.


We postulate that this approach to science may be a powerful tool to foster interdisciplinary collaborations, if used in a reflexive way in the construction of projects. More precisely, an ``Applied Perspectivism'' would imply a high-level of reflexivity for each agent implied, a mapping of the different layers of the enterprise and the positioning of each agent regarding the domains of knowledge. This way, in the particular case of model coupling, the explicitation of positioning and of the structure of each knowledge implied should ease interactions. As Banos points out~\cite{banos2013pour}, transversal work must alternate with deeper investigations in each discipline, in a kind of ``virtuous circle''. This raises the issue of, before individual researcher particularities, how a given collective structure of scientific knowledge production should balance between these two. It is clear that this question is deeply endogenous to each studied subject, and even each particular approach taken, but within the applied knowledge framework described above, we have reasons to believe that certain structural properties may be rather general. Indeed, each discipline is expected to bring components for each knowledge domain, and the co-evolving perspective is built on their interrelations. We propose to investigate here basic aspects of this issue, with means of agent-based modeling.


We aim at providing quantitative evidence of the feasibility of the epistemological point of view described above and inform potential implementation for some of its processes, more precisely how can certain level of coupling of perspectives (or overlap of ontologies) may be achieved given specializations of scientists and a given dynamic of interaction.



\subsection*{An agent-based Model of Interdisciplinarity}

Various works have dealt with microscopic modeling of knowledge production, among which for example Chavalarias' Nobel game~\cite{chavalarias2016s} which investigates the balance between falsification of previous theories and the elaboration of new theories. Giere also introduced an agent-based model of science in~\cite{giere2010agent}, consistently with the perspectivist approach described above. We develop here a simple agent-based model of scientific research focusing on the interplay between disciplinary and interdisciplinary research. The rationale relies on the basic assumption that scientists can choose when starting a new project between interdisciplinary collaboration and a work within their discipline. How can the choice patterns at the micro-level influence the overall interdisciplinarity level ? The model is voluntary parcimonious to test if even many simplification some structural effects still hold.


\paragraph{Model}

Agents are $N$ scientists $A_i$, characterized by a probability distribution $d(x)$ representing their disciplinary positioning in an abstract way: research is summarized by a one dimensional variable $\mathbb{R}$, and the disciplinary positioning on this axis is given by the distribution. The model is setup with normal distributions of width $\sigma$ with an average distributed uniformly in $\left[0;1\right]$. Scientists also have a time budget per day, that we will summarize as a future timetable $T(t_0):t>t_0 \mapsto p(t) \in \mathcal{P}$ where $\mathcal{P}$ is the space of scientific projects. The central feature of the model is the utility function $U(d_i,d_j)$ determining an abstract utility for scientist $i$ to collaborate with $j$ for a given project. It will be a function of the disciplinary overlap $o = \int_x d_i(x)\cdot d_j(x) dx$ and different assumptions on the form of this cost function can be tested. We take a linear cost in the overlap and a varying benefit, expressing the fact that researchers have different strategies regarding their interdisciplinary positioning. This way, we have $U(d_i,d_j) = o / i^\alpha - o$, assuming a fat-tail distribution of individual preferences for interdisciplinarity, given by a power law of parameter $\alpha$. A discrete choice formulation gives the probabilities for a scientist $i$ to choose among $j$ collaborators by $p_j = \exp\left(\beta U(d_i,d_j) \right)/\sum_k \exp\left(\beta U(d_i,d_k) \right)$. Given a social network of relations, that we take for now as a fixed scale-free social network, the temporal evolution of the model goes as follows: (i) one scientist with no current activity is picked up at random, and starts a project with one of its potential collaborators taken as its neighbors in the network that have free time, chosen with the probability $p_j$. The project has a random uniform duration and timetables are updated accordingly; (ii) current projects are updated and finished if necessary. The outcome of the model if measured by average depth across project, defined for one project as the overlapping areas between distribution, and average interdisciplinarity measured by total area covered.


\paragraph{Results}

The model is implemented in NetLogo and explored with OpenMole~\cite{reuillon2013openmole}. Source code and results are available on the open repository of the project\footnote{on github at \texttt{https://github.com/JusteRaimbault/Perspectivism}}. We run a basic grid exploration of the parameter space, both with random and small-world social networks, for parameters $\alpha,\beta,\sigma$ with 50 repetitions of the model for each parameter points, corresponding to 158,400 model runs. Figure~\ref{fig:plots} shows indicators variation on a given subspace and the corresponding Pareto front between depth and interdisciplinarity. We show a second order influence of preference hierarchy $\alpha$ and non-linearity of model behavior as a function of all parameters. Convergence properties are reasonable with this number of repetitions. Large individual disciplinary width $\sigma$ causes the choice parameter $\beta$ to have no influence, whereas low values give an increasing interdisciplinarity and a decreasing depth as a function of $\beta$. Random behavior ($\beta = 0$) leads to a constant depth of projects. When examining the Pareto front between the two contrary objectives, the optimal points occur for intermediate $\beta$ when $\sigma$ is fixed, suggesting non-trivial behavioral optima at a fixed disciplinary configuration. These first exploration show the complex dynamics of interdisciplinarity even with simple interaction rules and network structure, and suggests further applications such as the exploration of policies by changing network structure or studying in a more refined way the influence of $\alpha$.


\begin{figure*}
	\centering
	\includegraphics[width=0.32\linewidth]{figures/depth-sigma_alpha0-5_rho0}
	\includegraphics[width=0.32\linewidth]{figures/interdisc-sigma_alpha0-5_rho0}
	\includegraphics[width=0.32\linewidth]{figures/pareto_alpha0-5_rho0}
	\caption{\textbf{Patterns of interdisciplinarity from model simulations.} We show measures of depth and interdisciplinarity (left and middle) at fixed $\alpha=0.5$ and network structure, for varying discrete choice parameter $\beta$ as a function of individual extent $\sigma$. On the right, the Pareto front of average point between these two objectives.}
	\label{fig:plots}
\end{figure*}





\paragraph{Possible extensions}

Possible refinements of the model, towards a less stylized and more behavioral and micro-based model, could for example include the introduction of time budgets, simultaneous projects and dynamical time investment for scientists. The assumption of two-person projects is also strongly constraining, and relaxing it would require the extension of depth and interdisciplinarity measures that is not necessary straightforward. Furthermore, the absence of learning and of evolution of the social network when completing a project suggests a short time scale of application: further refinements should include dynamics of individual distributions and of individual relationships.


\bigskip

In conclusion, we show with this simple model that the individual choices produce an emerging structure of the research front, suggesting that applied perspectivism requires a careful tuning of research structure and researcher behaviors since Pareto-optimal configurations correspond to non-trivial parameter points. Future developments should include more realistic behavioral assumption, and a formalisation of the applied perspectivism approach to include it in the agent-based model.




%%%%%%%%%%%%%%%%%%%%
%% Biblio
%%%%%%%%%%%%%%%%%%%%
\footnotesize

\bibliographystyle{apalike}
\bibliography{/Users/Juste/ComplexSystems/CityNetwork/Biblio/Bibtex/CityNetwork,biblio}


\end{document}
