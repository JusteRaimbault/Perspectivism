%%%%%%%%%%%%%%%%%%%%%%%%%%%%%
% Standard header for working papers
%
% WPHeader.tex
%
%%%%%%%%%%%%%%%%%%%%%%%%%%%%%

\documentclass[10pt]{article}

%%%%%%%%%%%%%%%%%%%%
%% Include general header where common packages are defined
%%%%%%%%%%%%%%%%%%%%



%%%%%%%%%%%%%%%%%%%%%%%%%%
%% TEMPLATES
%%%%%%%%%%%%%%%%%%%%%%%%%%


% Simple Tabular

%\begin{tabular}{ |c|c|c| } 
% \hline
% cell1 & cell2 & cell3 \\ 
% cell4 & cell5 & cell6 \\ 
% cell7 & cell8 & cell9 \\ 
% \hline
%\end{tabular}





%%%%%%%%%%%%%%%%%%%%%%%%%%
%% Packages
%%%%%%%%%%%%%%%%%%%%%%%%%%


% general packages without options
\usepackage{amsmath,amssymb,bbm}

% graphics
\usepackage{graphicx}

% text formatting
\usepackage[document]{ragged2e}
\usepackage{pagecolor,color}








%%%%%%%%%%%%%%%%%%%%%%%%%%
%% Maths environment
%%%%%%%%%%%%%%%%%%%%%%%%%%

%\newtheorem{theorem}{Theorem}[section]
%\newtheorem{lemma}[theorem]{Lemma}
%\newtheorem{proposition}[theorem]{Proposition}
%\newtheorem{corollary}[theorem]{Corollary}

%\newenvironment{proof}[1][Proof]{\begin{trivlist}
%\item[\hskip \labelsep {\bfseries #1}]}{\end{trivlist}}
%\newenvironment{definition}[1][Definition]{\begin{trivlist}
%\item[\hskip \labelsep {\bfseries #1}]}{\end{trivlist}}
%\newenvironment{example}[1][Example]{\begin{trivlist}
%\item[\hskip \labelsep {\bfseries #1}]}{\end{trivlist}}
%\newenvironment{remark}[1][Remark]{\begin{trivlist}
%\item[\hskip \labelsep {\bfseries #1}]}{\end{trivlist}}

%\newcommand{\qed}{\nobreak \ifvmode \relax \else
%      \ifdim\lastskip<1.5em \hskip-\lastskip
%      \hskip1.5em plus0em minus0.5em \fi \nobreak
%      \vrule height0.75em width0.5em depth0.25em\fi}



%%%%%%%%%%%%%%%%%%%%
%% Idem general commands
%%%%%%%%%%%%%%%%%%%%
%% Commands

\newcommand{\noun}[1]{\textsc{#1}}


%% Math

% Operators
\DeclareMathOperator{\Cov}{Cov}
\DeclareMathOperator{\Var}{Var}
\DeclareMathOperator{\E}{\mathbb{E}}
\DeclareMathOperator{\Proba}{\mathbb{P}}

\newcommand{\Covb}[2]{\ensuremath{\Cov\!\left[#1,#2\right]}}
\newcommand{\Eb}[1]{\ensuremath{\E\!\left[#1\right]}}
\newcommand{\Pb}[1]{\ensuremath{\Proba\!\left[#1\right]}}
\newcommand{\Varb}[1]{\ensuremath{\Var\!\left[#1\right]}}

% norm
\newcommand{\norm}[1]{\| #1 \|}


\newcommand{\indep}{\rotatebox[origin=c]{90}{$\models$}}

%% graphics

% renew graphics command for relative path providment only ?
%\renewcommand{\includegraphics[]{}}







\usepackage[utf8]{inputenc}
\usepackage[T1]{fontenc}





% geometry
\usepackage[margin=3cm]{geometry}

% layout : use fancyhdr package
%\usepackage{fancyhdr}
%\pagestyle{fancy}

\makeatletter

%\renewcommand{\headrulewidth}{0.4pt}
%\renewcommand{\footrulewidth}{0.4pt}
%\fancyhead[RO,RE]{\textit{Working Paper}}
%\fancyhead[LO,LE]{G{\'e}ographie-Cit{\'e}s/LVMT}
%\fancyfoot[RO,RE] {\thepage}
%\fancyfoot[LO,LE] {\noun{J. Raimbault}}
%\fancyfoot[CO,CE] {}

\makeatother


%%%%%%%%%%%%%%%%%%%%%
%% Begin doc
%%%%%%%%%%%%%%%%%%%%%

\begin{document}







\title{Applied Perspectivism to Foster Interdisciplinarity\\\medskip
\textit{Ecole thématique de Rochebrune 2018}
}
\author{\noun{Juste Raimbault}$^{1,2}$\\
$^{1}$ UMR CNRS 8504 Géographie-cités\\
$^{2}$ UMR-T IFSTTAR 9403 LVMT
}
\date{}


\maketitle

\justify

\begin{abstract}
	
\end{abstract}




\subsection*{Perspectives and Ontologies}

Beyond the simplifying opposition between fully constructivist and realistic approaches to science, several alternative have been developed, among which Perspectivism \cite{giere2010scientific} is a way to tackle most of the issues opposing these two by taking an agent-based approach to the production of scientific knowledge. 




\subsection*{Applied Perpectivism and Model Coupling}


We postulate that this approach to science may be a powerful tool to foster interdisciplinary collaborations, if used in a reflexive way in the construction of projects. More precisely, an ``Applied Perspectivism'' would imply a high-level of reflexivity for each agent implied, and a mapping of the different layers of the entreprise and the positioning of each agent regarding the domains of knowledge.




\subsection*{An agent-based Model of Interdisciplinarity}

\paragraph{Model}

The second part of this paper aims at providing quantitative evidence of the feasibility of the epistemological point of view described above and inform potential implementation for some of its processes. Therefore, we introduce a simple agent-based model of scientific research focusing on the interplay between disciplinary and interdisciplinary research. The rationale relies on the basic assumption that scientists can choose when starting a new project between interdisciplinary collaboration and a work within their discipline. How can the choice patterns at the micro-level influence the overall interdisciplinarity level ?

Agents are $N$ scientists $A_i$, characterised by a probability distribution $d(x)$ representing their disciplinary positioning in an abstract way. They also have a time budget 


\paragraph{Results}




\paragraph{Possible extensions}

Possible refinements of the model, towards a less stylized and more behavioral and micro-based model, could for example include the introduction of time budgets, simultaneous projects and dynamical time investment for scientists. The assumption of two-person projects is also strongly constraining, and relaxing it would reauire the extension of depth and interdisciplinarity measures that is not necessary straightforward. Furthermore, the absence of learning when completing a project suggests a short time scale of application.





%%%%%%%%%%%%%%%%%%%%
%% Biblio
%%%%%%%%%%%%%%%%%%%%
\footnotesize

\bibliographystyle{apalike}
\bibliography{/Users/Juste/ComplexSystems/CityNetwork/Biblio/Bibtex/CityNetwork,biblio}


\end{document}
